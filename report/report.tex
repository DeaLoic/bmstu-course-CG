\documentclass{article}

\usepackage{GOST}
\usepackage[T1, T2A]{fontenc}
\usepackage[utf8]{inputenc}
\usepackage[russian]{babel}
\usepackage{cmap}
\usepackage{amssymb}
\usepackage{amsmath}
\usepackage{hyperref}

\usepackage{listings}
% Для листинга кода:
\lstset{ 
	language=C,                 % выбор языка для подсветки (здесь это С)
	basicstyle=\small\sffamily, % размер и начертание шрифта для подсветки кода
	numbers=left,               % где поставить нумерацию строк (слева\справа)
	numberstyle=\tiny,           % размер шрифта для номеров строк
	stepnumber=1,                   % размер шага между двумя номерами строк
	numbersep=5pt,                % как далеко отстоят номера строк от подсвечиваемого кода
	showspaces=false,            % показывать или нет пробелы специальными отступами
	showstringspaces=false,      % показывать или нет пробелы в строках
	showtabs=false,             % показывать или нет табуляцию в строках
	frame=single,              % рисовать рамку вокруг кода
	tabsize=2,                 % размер табуляции по умолчанию равен 2 пробелам
	captionpos=t,              % позиция заголовка вверху [t] или внизу [b] 
	breaklines=true,           % автоматически переносить строки (да\нет)
	breakatwhitespace=false, % переносить строки только если есть пробел
}


\graphicspath{{images/}}

\linespread{1.5}

\begin{document}
	\input{0_title.tex}
	\newpage
	\tableofcontents
	\newpage
	\begin{center}
	    \section*{Введение}
	\end{center}
	\addcontentsline{toc}{section}{Введение}
	\indent \indent Компьютерные системы уже глубоко проникли во все сферы жизни и являются неотъемлемыми составляющими все более различной человеческой деятельности. Организация работы предприятий, проектирование ракет, моделирование химических процессов - все это значительно облегчилось после широкого распространения компьютеров.
		\newline
	\indent При все большем усложнении информационных систем и развитии компьютерной техники, росли и требования к таким системам. Очень быстро появилась потребность в визуализации данных, полученных или обрабатанных с помощью уже существующего программного обеспечения. Ответом на эту потребность стала машинная графика - область компьютерной науки, отвечающая за обработку, синтез и распознавание изображений. В частности выделилось такое направление машинной графикиg, как 3D-моделирование, отвечающее за синтез и обработку изображений объемных объектов.
\newline
	\indent В настоящее время существует большое количество задач, решаемых с помощью 3D-моделирования, например высокоточное моделирование деталей и объектов, добавление спецэффектов при производстве фильмов, компьютерные игры. Одной из таких задач является генерация трехмерного ландшафта.
		
	\newpage
	\section{Аналитическая часть}
	В данном разделе будут поставлены задачи работы, будут рассмотренны основные теоритические сведения связанные с трехмерной генерацией ландшафта.
	\\ \indent
	Задачу генерации трехмерного ландшафта можно решать различными способами, но почти все из них можно разделить на следующие этапы:
	\begin{enumerate}
		\item генерация карты высот;
		\item построение трехмерного изображения по карте высот;
		\item текстурирование.
	\end{enumerate}
	Этап текстурирования может быть опущен.

	\newpage
	\section{Конструкторская часть}
	
	\newpage
	\section{Технологическая часть}
	

	\newpage
	\section{Экспериментальная часть}
	
	\subsection{Вывод}

	\newpage
	\begin{center}
		\section*{Заключение}
	\end{center}
	\addcontentsline{toc}{section}{Заключение}
	\newpage
	\addcontentsline{toc}{section}{Список литературы}
	
	\begin{center}
	\begin{thebibliography}{3}
	\bibitem{binary-search}
	Целочисленный двоичный поиск. ITMO [Электронный ресурс]. Режим доступа: (дата обращения - 20.11.2020) Свободный. URL: https://neerc.ifmo.ru/wiki/index.php?title=Целочисленный\_двоичный\_поиск
	\bibitem{c-sharp}
	Golang [Электронный ресурс]. Режим доступа: (дата обращения - 20.11.2020) Свободный. URL: https://ru.wikipedia.org/wiki/C\_Sharp
	\bibitem{vs}
	Visual Studio [Электронный ресурс]. Режим доступа: (дата обращения - 20.11.2020) Свободный. URL: https://visualstudio.microsoft.com/ru/

	\end{thebibliography}
	\end{center}
\end{document}